\chapter{Spatial Kappa simulator User Guide}

\section{Obtaining the simulator}

The simulator is available from GitHub as the source Eclipse project, or as a single executable jar file. Both are available at \url{https://github.com/donal-s/SpatialKappa/downloads}. The simulator in its current state was optimised specifically for the 2010 Edinburgh iGEM project. This is apparent mostly in the design of the compartment view pane. Extending this view for more generic use is a planned extension.

\section{Starting the simulator}

\subsection{Running the executable jar}

The simulator can be started by running the executable jar file:\\
\verb|java -jar SpatialKappa-v1.0.0.jar|\\\\
Double clicking the jar file usually works too.

\subsection{Running from the Eclipse project}

The main class of the simulator is \\
\verb|org.demonsoft.spatialkappa.ui.TransitionMatchingSimulatorGui|\\\\
Running as a Java Application will bring up the simulator.


\section{Using the simulator}


The initial screen appears as figure \ref{fig:startup}.

\begin{figure}[h!]
 \centering
 \includegraphics[scale=0.3]{./images/startup.png}
 % startup.png: 1006x510 pixel, 72dpi, 35.49x17.99 cm, bb=0 0 1006 510
 \caption{Initial view}
 \label{fig:startup}
\end{figure}

The toolbar options are:

\begin{tabular}{cl}
 \includegraphics[scale=0.5]{./images/Open.png} & Open a Kappa or Kappa replay file \\
 \includegraphics[scale=0.5]{./images/Reopen.png} & Reopen the current file  \\
 \includegraphics[scale=0.5]{./images/Run.png} & Run the current file \\
 \includegraphics[scale=0.5]{./images/Replay.png} & Replay the current replay file or last run \\
 \includegraphics[scale=0.5]{./images/Stop.png} & Stop the currently running simulation \\
 \includegraphics[scale=0.5]{./images/EventsOrTime.png} & Choose event or time based simulation \\
\end{tabular}

\subsection{Opening a Kappa or Kappa replay file}

Select the 'Open' button on the toolbar and select the file to open. The current implementation expects Kappa source files to have the suffix \verb|.ka| and Kappa replay files (discussed later) to have the suffix \verb|.kareplay|. If the file is parsed successfully, a summary of the Kappa model is displayed in the 'Data' pane (see figure \ref{fig:dataPane}). 

\begin{figure}[h!]
 \centering
 \includegraphics[scale=0.3]{./images/FileOpenDataPane.png}
 % FileOpenDataPane.png: 1006x510 pixel, 72dpi, 35.49x17.99 cm, bb=0 0 1006 510
 \caption{Data pane showing loaded Kappa model}
 \label{fig:dataPane}
\end{figure}

Any errors in reading the Kappa file are shown in the 'Console Output' pane.

The currently open Kappa file can be refreshed from disk by selecting the 'Reopen' button. Useful when editing the Kappa model.

\subsection{Running a simulation}

With a successfully opened Kappa model, one can run a simulation by selecting the 'Run' button. Simulation parameters can be set on the toolbar before running. There is the option to do an event or a time based simulation. For an event based simulation, the number of steps for the simulation (i.e. data points on the time series chart), and the number of finite rate events per step can be set. Equivalent options for time based simulation can also be set.

The simulation can be halted at any point by selecting the 'Stop' button. Note that complex simulations may take a while to start up while data structures are being generated.

\subsection{Running a simulation replay}

As the simulation runs, the state of the simulation observables are logged to disk in a replay file after every step. Once the simulation is complete, this replay file can be rerun by selecting the 'Replay' button. The 'Replay Interval' field allows a delay (in ms) to be added between each step.

Note - the current storage format is binary, and version dependent. Creation of a more permanent trace storage format is a planned enhancement.


\section{Spatial visualisation tools}

While the raw data produced from simulations is useful, visualisation of the data is important. There are a couple of simple visualisation panels in the simulator. These are dynamically updated as the simulation runs to give the user an idea of how the simulation is progressing. They are however basic in comparison to some of the commercial simulation data visualisation tools available.

The excellent JFreeChart \citep{JFreeChartwebsite} library was used for generating the charting components. Both charts have formatting and save capability, and the time series panel is zoomable.

\subsection{Time series chart}

This chart is similar to the standard Gnuplot output from Simplx. It is a line graph showing observable quantity against time for all observable definitions in the model.

\begin{figure}[h!]
 \centering
 \includegraphics[scale=0.3]{./images/ChartPane.png}
 % ChartPane.png: 1006x510 pixel, 72dpi, 35.49x17.99 cm, bb=0 0 1006 510
 \caption{Sample time series chart output}
 \label{fig:chartPane}
\end{figure}

\subsection{Compartment chart}

This view allows more detailed visualisation of the transport of a species within the cells of a single compartment. It can show local positive feedback sites, or diffusion of a species through a compartment. There are a couple of visualisation options which can be selected at runtime to tailor the output.

Note that the current implementation is designed to view a 3-channel hexagonal mesh, and that view works only for observations labelled \verb|'Red'|, \verb|'Blue'| and \verb|'Green'| in a compartment labelled \verb|'Cytosol'|. Making this more generic is also a planned area for extension.

\begin{figure}[h!]
 \centering
 \includegraphics[scale=0.3]{./images/CompartmentView.png}
 % CompartmentView.png: 1006x510 pixel, 72dpi, 35.49x17.99 cm, bb=0 0 1006 510
 \caption{Sample compartment chart output}
 \label{fig:compartmentPane}
\end{figure}
